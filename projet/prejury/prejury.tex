TODO
\ documentclass { article }
% Calcul de la fréquence de coupure pour un circuit passe-haut via l'intersection de deux droites représentant 
% la variation de tension de sortie en foction de la fréquence du signal entrant.

\ title { Evaluation de la fréquence de coupe d'un filtre passe-haut pour une résistance de \dots et une capacité de ... }
\ author { Groupe 11.53 }
\ date { \today }

\usepackage[french]{babel}
\usepackage [ latin 1]{ inputenc }
\usepackage [T1]{ fontenc }
\usepackage { lmodern }

\begin { document }

\maketitle 

Dans cette partie, nous allons vous expliquer comment , grâce à des données expérimentales
obtenues durant nos laboratoires du lundi après-midi, à l'aide d'un générateur de fonction, d'un oscilloscope et,
bien sur, un circuit R-C,résistance-capacité, servant de filtre passe bas dans cette section, nous avons calculé la 
fréquence de coupure de notre filtre.   

   A chaque fréquence que générenotre générateur de fonction correspond une tension, correspondant à celle au bornes
   de la capacité, que nous pouvons observer sur l'oscilloscope. En dessinant le graphe de la tension $V_c$ aux bornes de 
   a capacité, on voit que l'ensemble des points se regroupent autour d'une certaine droite d'équation \begin { equation }
   y = m * x + p \end { equation }, où m est la pente, p l'ordonnée à l'origine et $$\ left (\ begin { array }{ cc}
x\\
y 
\end{ array }\ right )$$
est un point appartenant à la droite. Cette droite est croissante jusqu'à ce qu'elle atteigne une certaine fréquence,
à laquelle elle se stabilise. Petite indiction pratique, nous avons mis l'axe "x" des abscisses dans un repère logarithmique
de base 10. Cela signifie quà la place d'avoir à chaque graduation 0,1,2,3, ... Hz , nous avons $10^0$, $10^1$, $10^2$,... Hz.
Ceci nous permet de mieux voir notre graphe car commençons par des fréquences de quelques dizaines de Hz pour monter dans les 
fréquencez de plusieurs kHz.
Nous calculons maintenant les inconues m et p de l'équation de notre première droite, croissante, au moyen d'un système 
 \begin {equation} A * $\overrightarrow{x}$ = $\overrightarrow{b}$ \end {equation}, où A = $$\ left (\ begin { array }{ cc}
log($f_1$) & 1 \\
log($f_2$) & 1 \\
log($f_3&) & 1 \\
\end{ array }\ right )$$

,$\overrightarrow{x}$ = $$\ left (\ begin { array } { cc} m \\ p \end { array }\ right )$$ et $\overrightarrow{b}$ = 

$$\ left (\ begin {array } { cc} $V_{c1}$ \\ $V_{c2}$ \\ $V_{c3}$ \end{array }\ right )$$ .

Avec données, cela devient:

\begin {equation}  $$\ left (\ begin { array }{ cc} 127 & 1 \\ 191 & 1 \\ 356 & 1 \end{ array }\ right )$$ * 
$$\ left (\ begin { array }{ cc} m \\ p \end{ array}\right )$$ = $$\ left (\ begin { array }{ cc} 0,40 \\ 0,50 \\ 0,60 \end{ array}\right )$$ 




 

\end { document }
